\documentclass[11pt,letterpaper,]{article}

% AMS packages.
\usepackage{amsmath,amsthm,amsfonts,mathtools,amssymb}
\usepackage{booktabs}
\usepackage{cite}
\usepackage{float}
\usepackage{hyperref}
\usepackage{cleveref}

\newtheorem{theorem}{Theorem}

\title{Verifying Burrows Wheeler Transformation In Coq}
\author{Hanlin He, Sourav Dasgupta, Paul Parrot, Alan Padilla Chua}

\begin{document}

\maketitle

\section{Motivation}

The idea for this research project was given to our team by Dr. Kevin Hamlen.
Our initial goal was to verify a full compression algorithm from top to bottom.
However, given time constraints we narrowed our search to \emph{bzip2}'s
compression stack and saw that its core algorithm depended on the Burrows
Wheeler Transformation\cite{bzip2}. BWT (Burrows Wheeler Transformation) is a
block-sorting compression algorithm designed to improve the compressibility of
a given string. BWT also has some important security properties related to
integrity. It has a lossless property, meaning the forward transformation can
be reversed to get back the original string. This means that there is no loss
of data in either the forward or inverse transformations. We saw BWT to be of
great potential to be verified in Coq. Its full verification could narrow the
trusted code base of \emph{bzip2}'s compression and allow future teams to
tackle the full verification of \emph{bzip2}.

\section{Preliminaries}

Our Coq implementation has been tested on the CoqIDE version 8.6.1 for Windows,
Mac, and Linux.

\section{Project Summary}

\subsection{Approach}

For our implementation of BWT in Coq we decided to prove that the
transformation is lossless; Meaning that for any string we apply the forward
BWT on, we can subsequently perform the inverse BWT on that result and get the
original string back. In order to accomplish this and be able to reason about
it, we implemented the forward BWT and inverse BWT in Coq. Before getting to
these two implementations we need to discuss some more general, but still
important, parts of our implementation.

Dr. Kevin Hamlen mentioned that proving theorems about lists could lead to
complex issues. So instead of using lists, for our main implementation we chose
to implement strings as mappings ($str: nat \rightarrow option\ nat$) and
matrices as a mapping of mappings ($matrix: nat \rightarrow str$). It is worth
noting that we did create a generic implementation of BWT using lists but did
not attempt to prove any lemmas about it. Our proofs are solely related to the
string mapping version of our Coq code.

Following is a description of our approach to implementing the forward BWT.
Given a string $S$ as input, we first convert the string into a $str$ mapping
$s$. Then we construct a matrix by right shifting $s$. This fills the matrix
with each permutation of the original string. To avoid proving a sort function
for our mapping, we then assume a sort function is given that sorts the matrix
lexicographically. Finally, we return the last column of the sorted matrix as
the result string.

Following is a description of our approach to implementing the inverse BWT.
Given a string $L$, an index $i$, and a sort function as input, we first sort
$L$ lexicographically. Then we generate the index mapping $\pi_L$ between the
given string and the sorted string. We then recursively generate a sequence of
indices as $\pi_L^k(i)$, each of which corresponds to the $k$-th character in
the original string. Finally, we apply all the indices with the given string
$L$ to regenerate the original string before the forward BWT.

\subsection{Difficulties}

\subsubsection{Sorting}

The BWT algorithm requires two stable lexicographical sorts: matrix sort and
string sort. Our main implementation of strings and matrices uses an abstract
mapping of the strings rather than lists. This prohibited us from importing a
Coq sorting library because it was unclear how our abstract mapping could use a
sorting library meant for lists. To overcome this, we decided that instead of
proving a sorting algorithm ourselves, we would assume either a sorting
algorithm that satisfies a set of properties for our abstract mapping is given,
or the sorted version of the string or matrix mapping is given.

\subsubsection{Understanding the Mathematical Paper Proof of Burrows Wheeler
Transform}

Before we go into the paper proof, we would like to point out that the proof
uses the following notations:

\begin{itemize}
    \item $\Sigma$: Finite, non-empty symbol/alphabet set. Every string $L$ is
        composed of a sequence of symbols which belong to $\Sigma$.
    \item $\lambda_L$: A mapping $\mathbb{N}\rightarrow\Sigma$, which given a
        natural number $i \leq length(L)$, will return the symbol $\alpha$
        located at the $i$th position in $L$.
    \item $\pi_L$: A mapping $\mathbb{N}\rightarrow\mathbb{N}$, which maps an
        index $i \leq length(L)$ with the index of the symbol $\alpha$ in $L$
        that should have been in position $i$ if the symbols in string $L$ were
        stable sorted.
    \item $context_k(L)$: Means the $k$ length prefix of string $L$.
    \item $\textit{conjugacy class of string }L$: A sequence or collection of
        strings where each string $w_i$ is a right-shift of $w_{i-1}$, with
        $w_0 = L$. Note that the right shift is a circular one, i.e., the last
        symbol in $w_{i-1}$ goes to the beginning of $w_i$ . We treat the
        conjugacy class of $L$ as a matrix.
\end{itemize}

The preceding notations have been described very briefly for convenience. For
full definitions, the reader is requested to refer\cite{kufleitner2009on}.

The paper theorem follows:

The statement is a generic one which states that, given a matrix of $k$ order
context sorted conjugacy class of a string $m$ , the string $L$ composed of the
last letters of each of the string in the sorted conjugacy class, the $t$-order
context of $w_i$ can be constructed as follows: starting with $i$ , and
recursively apply the $\pi_L$ function, and take the letter in $L$ , indexed at
each recursive application of $\pi_L$. The recursion continues till $k$.

A more formal way of presenting the above stated notion (as mentioned in Lemma
3 of \cite{kufleitner2009on}) is as follows:

\begin{theorem}
\newcommand*{\context}[2]{\ensuremath{\mathrm{context}_{#1}(#2)}}
\newcommand*{\last}[1]{\ensuremath{\mathrm{last}(#1)}}
Let $k \in \mathbb{N}$. Let
$\bigcup_{i=1}^s[v_i]=\{w_1,\ldots,w_n\}\subseteq\Sigma^+$ be a multiset built
from conjugacy classes $[v_i]$. Let $M=(w_1,\ldots,w_n)$ satisfy
$\context{k}{w_1}\leq\cdots\leq\context{k}{w_n}$ and let
$L=\last{w_1}\cdots\last{w_n}$ be the sequence of the last symbols.

Then
\[\context{k}{w_i}=\lambda_L\pi_L(i)\cdot\lambda_L\pi_L^2(i)\cdots\lambda_L\pi_L^k(i)\]
where $\pi_L^k$ denotes the $t$-fold application of $\pi_L$ and $\lambda_L\pi_L^t(i)=\lambda_L\big(\pi_L^t(i)\big)$.
\end{theorem}

The proof is done using induction. While we do not present the entire proof in
the paper, the core idea is briefly explained as follows: It is trivial to show
that the above theorem holds for $k=0$. As the induction hypothesis, it is
assumed in the paper that the for any $t \leq k$, the theorem holds for $t-1$.
Using the hypothesis, each $w_i$ is then right shifted and sorted by $t$-order
context resulting in the sequence $(u_1,\ldots,u_n)$. Following this, it is
argued that $t$-order context of $u_i$ equals the $t$-order context of
$w_{\pi_L}(i)$. Plugging this relation to the induction hypothesis yields the
induction.

\subsubsection{List of Necessary Implications for the Final Theorem}

Interestingly, the proof mentioned in the previous section itself tells us how
to compute the inverse of BWT. If we have the string $L$ (which is the BWT
transform of $m$) and $\pi_L$ permutation (which can be constructed from $L$ by
sorting $L$ ), we can compute $k$ order context of any $w_i$ belonging to the
conjugacy class of our original string $m$. Note that since $m$ itself is one
of the $w_i$'s, if we start with the correct value of $i$, we can get the
original string $m$. Also note that if we set the value of $k$ to be
$length(m)$, we get the entire $m$ back.

\section{Accomplishments}

\subsection{Theorem}

\emph{Matrix} is the theorem that we constructed from our implementation that
we believe proves the losslessness of the BWT. This theorem states that given
the last column $L$ (and the sorted last column $L\_sorted$), of the sorted
right shift matrix, and an index $i$, we can perform the inverse BWT which will
return the string at index $i$ in the sorted right shift matrix. This theorem
states a general solution whereby we can reconstruct any string from the
conjugacy matrix given the forward BWT result string and the index at which
that string resides within the matrix. Obviously this implies that with our
solution we can also reconstruct the original string that was given so long as
we know its index.

\subsection{Working Functions (Definitions)}

\subsubsection{\texttt{bwt}}

This function is our implementation of the forward BWT. It takes a string and a
sort function as input. First, it creates a string mapping from the given
string. Then it generates a matrix such that each row contains one right
shifted permutation of the string mapping. Then it uses the sort function to
lexicographically sort the rows. The last column, denoted $L$, is the output
string. This implementation relies on several helper functions described below.

\begin{enumerate}
    \item \verb|string_to_map|\\
        This function takes a normal ascii string and returns it in a mapping
        form that is defined by us. This mapping makes it easier for us to
        generate a matrix of permutations for the string.
    \item \verb|map_to_conjugacy|\\
        This function returns a matrix such that each row contains a
        permutation of the original string. These permutations are all right
        shifted by one character and wrap around to the beginning of the row.
    \item \verb|map_to_string|\\
        This function takes our defined mapping form of a string and returns
        the corresponding string.
\end{enumerate}

\subsubsection{\texttt{inverse\_bwt}}

This function is our implementation of the inverse BWT. It simply applies
\verb|lambda| to each index in the string returned by \verb|inverse_bwt'|; The
result of which is the original string. This implementation relies on several
helper functions described below.

\begin{enumerate}
    \item \verb|pi|\\
        This function takes a BWT string, and it's lexicographically sorted
        permutation, as input and returns a mapping. The returned mapping
        represents the index of each letter in the transformed string.
    \item \verb|inverse_bwt'|\\
        This function uses \verb|pi| to generate the index mapping. It then
        recursively generates a sequence of indices; Each of which corresponds
        to one character in the original string.
    \item \verb|lambda|\\
        This function takes an input string $m$ and index $i$ and returns the
        character at position $i$ in $m$.
\end{enumerate}

\section{Team Coordination}

\begin{table}[H]
\centering
\caption{Coordination}
\begin{tabular}{l|r}
\toprule
Task & Person\\
\midrule
Presentation Slides & Alan, Paul, Hanlin, Sourav\\
Presentation & Alan, Hanlin\\
Coq Implementation & Hanlin, Sourav, Alan\\
Coq Proofs & Hanlin, Sourav, Paul\\
Report & Alan, Paul, Hanlin, Sourav\\
\bottomrule
\end{tabular}
\end{table}

\section{Future Work}

We laid a solid foundation for a proper proof to be completed. However certain
circumstances, described earlier in the report, prevented us from fully
achieving our goal. For future work we propose the full completion of the
theorem using our current implementation and potentially some modifications
that will come from working on the proof. We believe that completion of this
proof will allow future teams to focus on formally verifying the other sections
of \emph{bzip2}'s compression stack; And eventually leading to a complete,
machine-verified, proof of \emph{bzip2}.

\bibliographystyle{plain}
\bibliography{report.bib}{}

\end{document}

